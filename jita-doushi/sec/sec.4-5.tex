\section[派生形相互の対応]{派生形相互の対応\gosuu{\textbf{248}}}

\subsection{aru(自・ラ行五段)・eru(他・下一段)\gosuu{140}}
\nopagebreak
\begin{hyo}
    上&あがる&あげる&&&&& \\
    揚&あがる&あげる&&&&& \\
    挙&あがる&あげる&&&&& \\
    預&あずかる&あずける&&&&& \\
    暖&あたたまる&あたためる&&&&& \\
    温&あたたまる&あたためる&&&&& \\
    当&あたる&あてる&&&&& \\
    集&あつまる&あつめる&&&&& \\
    改&あらたまる&あらためる&&&&& \\
    受&うかる&うける&&&&& \\
    薄&うすまる&うすめる&&\kome うすれる&&& \\
    &&&&\kome うすらぐ&&& \\
    埋&うまる&うめる&&\kome うもれる&&& \\
    植&うわる&うえる&&&&& \\
    納&おさまる&おさめる&&&&& \\
    治&おさまる&おさめる&&&&& \\
    収&おさまる&おさめる&&&&& \\
    修&おさまる&おさめる&&&&& \\
    教&&おそわる&おしえる&&&&「他使の対応」、形態上ここに整理した。語根が異なる。 \\
    終&おわる&おえる&&&&& \\
    架&かかる&かける&&&&& \\
    懸&かかる&かける&&&&& \\
    掛&かかる&かける&&&&& \\
    重&かさなる&かさねる&&&&& \\
    固&かたまる&かためる&&&&& \\
    代&かわる&かえる&&&&& \\
    変&かわる&かえる&&&&& \\
    換&かわる&かえる&&&&& \\
    替&かわる&かえる&&&&& \\
    決&きまる&きめる&&&&& \\
    清&きよまる&きよめる&&&&& \\
    極&きわまる&きわめる&&&&& \\
    窮&きわまる&きわめる&&&&& \\
    加&くわわる&くわえる&&&&& \\
    下&さがる&さげる&&&&& \\
    授&&さずかる&さずける&&&&「他使の対応」、形態上ここに整理した。 \\
    定&さだまる&さだめる&&&&& \\
    静&しずまる&しずめる&&&&& \\
    鎮&しずまる&しずめる&&&&& \\
    閉&しまる&しめる&&&&& \\
    締&しまる&しめる&&&&& \\
    絞&しまる&しめる&&&&& \\
    狭&せばまる&せばめる&&&&& \\
    備&そなわる&そなえる&&&&& \\
    染&そまる&そめる&&&&& \\
    高&たかまる&たかめる&&&&& \\
    助&たすかる&たすける&&&&& \\
    携&たずさわる&たずさえる&&&&& \\
    縮&ちぢまる&ちぢめる&&ちぢむ&&&「ちぢむ」「ちぢめる」は、\kazu{\Rmnum{2}}の\kazu{1}\linebreak「ちぢむ」「ちぢまる」は、\kazu{\Rmnum{3}}の自・自 \\
    伝&つたわる&つたえる&&つたう&&&「つたう」「つたえる」は、\kazu{\Rmnum{2}}の\kazu{1}\linebreak「つたう」「つたわる」は、\kazu{\Rmnum{3}}の自・自 \\
    勤&つとまる&つとめる&&&&& \\ \pagebreak
    詰&つまる&つめる&&つむ&&&\SetCell[r=2]{l}「つむ」「つめる」は、\kazu{\Rmnum{2}}の\kazu{1}\linebreak「つむ」「つめる」は、\kazu{\Rmnum{2}}の自・自\linebreak「つむ」「つまる」は、\kazu{\Rmnum{3}}の自・自 \\
    &&&&つめる&&& \\
    強&つよまる&つよめる&&&&& \\
    連&つらなる&つらねる&&&\kome つれる&& \\
    止&とまる&とめる&&&&& \\
    泊&とまる&とめる&&&&& \\
    留&とまる&とめる&&&&& \\
    始&はじまる&はじめる&&&&& \\
    早&はやまる&はやめる&&&&& \\
    低&ひくまる&ひくめる&&&&& \\
    広&ひろまる&ひろめる&&&&& \\
    深&ふかまる&ふかめる&&&&& \\
    隔&へだたる&へだてる&&&&& \\
    曲&まがる&まげる&&&&& \\
    交&まじわる&まじえる&&&&& \\
    交&まざる&まぜる&&\kome まじる&&& \\
    混&まざる&まぜる&&\kome まじる&&& \\
    休&やすまる&やすめる&&やすむ&&&「やすむ」「やすめる」は、\kazu{\Rmnum{2}}の\kazu{1}\linebreak「やすむ」「やすまる」は、\kazu{\Rmnum{3}}の自・自 \\
    弱&よわまる&よわめる&&\kome よわる&&& \\
    分&わかる&わける&&わかれる&\kome わかつ&&「わかれる」「わける」は、\kazu{\Rmnum{4}}の\kazu{1}\linebreak「わかる」「わかれる」は、\kazu{\Rmnum{2}}の自・自 \\
\end{hyo}

\begin{reigai}
    分&わかれる&わける&&わかる&\kome わかつ&&「わかれる」は「わかる」から派生 \\
\end{reigai}

\subsection{eru(自・下一段)・asu(他・サ行五段)\gosuu{76}}

\begin{hyo}
    甘&あまえる&あまやかす&&&&&\\
    明&あける&あかす&&あく&あける&&「あく」「あける」は、\kazu{\Rmnum{2}}の\kazu{1}\linebreak「あく」「あかす」は、\kazu{\Rmnum{3}}の\kazu{1} \\
    荒&あれる&あらす&&&&& \\
    遅&おくれる&おくらす&&&&& \\
    枯&かれる&からす&&&&& \\
    腐&くされる&くさらす&&&くさる&&「くさる」「くさらす」は、\kazu{\Rmnum{3}}の\kazu{1}\linebreak「くさる」「くされる」は、\kazu{\Rmnum{2}}の自・自 \\
    暮&くれる&くらす&&&&& \\
    肥&こえる&こやす&&&&& \\
    焦&こげる&こがす&&こがれる&&&「こがれる」「こがす」は、\kazu{\Rmnum{1}}の\kazu{2}\linebreak「こげる」「こがれる」は、別語 \\
    転&ころげる&ころがす&&\kome ころぶ&&&「ころがる」「ころがす」は、\kazu{\Rmnum{1}}の\kazu{1}\linebreak「ころがる」「ころげる」は、\kazu{\Rmnum{4}}の自・自 \\
    &&&&ころがる&&& \\
    冷&さめる&さます&&&&& \\
    覚&さめる&さます&&&&& \\
    透&すける&すかす&&すく&&&「すく」「すかす」は、\kazu{\Rmnum{3}}の\kazu{1}\linebreak「すく」「すける」は、\kazu{\Rmnum{2}}の自・自 \\
    絶&たえる&たやす&&&\kome たつ&& \\
    垂&たれる&たらす&&&たれる&&「たれる」「たらす」は、\kazu{\Rmnum{4}}の他・他 \\
    縮&ちぢれる&ちぢらす&&&&& \\
    費&ついえる&ついやす&&&&& \\
    疲&つかれる&つからす&&&&& \\
    照&てれる&てらす&&てる&&&「てる」「てらす」は、\kazu{\Rmnum{3}}の\kazu{1}\linebreak「てる」「てれる」は、別語 \\
    出&でる&だす&&&&& \\
    解&とける&とかす&&&とく&&「とける」「とく」は、\kazu{\Rmnum{2}}の\kazu{2}\linebreak「とく」「とかす」は、\kazu{\Rmnum{3}}の他・他 \\
    溶&とける&とかす&&&とく&&「とける」「とく」は、\kazu{\Rmnum{2}}の\kazu{2}\linebreak「とく」「とかす」は、\kazu{\Rmnum{3}}の他・他 \\
    慣&なれる&ならす&&&&& \\
    逃&にげる&にがす&&のがれる&のがす&&「のがれる」「のがす」は、\kazu{\Rmnum{1}}の\kazu{2} \\
    抜&ぬける&ぬかす&&ぬかる&ぬく&&「ぬける」「ぬく」は、\kazu{\Rmnum{2}}の\kazu{2}\linebreak「ぬかる」「ぬく」は、\kazu{\Rmnum{3}}の\kazu{3}\linebreak「ぬく」「ぬかす」は、\kazu{\Rmnum{3}}の他・他\linebreak「ぬける」「ぬかる」は、\kazu{\Rmnum{4}}の自・自 \\
    生&はえる&はやす&&&&& \\
    化&ばける&ばかす&&&&& \\
    果&はてる&はたす&&&&& \\
    晴&はれる&はらす&&&&& \\
    冷&ひえる&ひやす&&&\kome ひやかす&& \\
    殖&ふえる&ふやす&&&&& \\
    増&ふえる&ふやす&&&&& \\
    更&ふける&ふかす&&&&& \\
    紛&まぎれる&まぎらす&&&\kome まぎらわす&& \\
    負&まける&まかす&&&&& \\
    蒸&むれる&むらす&&むす&むす&&「むれる」「むす」は、\kazu{\Rmnum{1}}の\kazu{2}\linebreak「むす」「むらす」は、\kazu{\Rmnum{3}}の\kazu{1}\linebreak「むす」「むらす」は、\kazu{\Rmnum{3}}の他・他\linebreak「むす」「むれる」は、\kazu{\Rmnum{1}}の自・自 \\
    燃&もえる&もやす&&&もす&&「もす」「もえる」は、\kazu{\Rmnum{1}}の\kazu{6}\linebreak「もす」「もやす」は、\kazu{\Rmnum{3}}の他・他 \\
    漏&もれる&もらす&&もる&&&「もる」「もらす」は、\kazu{\Rmnum{3}}の\kazu{1}\linebreak「もる」「もれる」は、\kazu{\Rmnum{2}}の自・自 \\
\end{hyo}

\subsection{iru(自・上一段)・asu(他・サ行五段)\gosuu{16}}

\begin{hyo}
    飽&あきる&あかす&&&&& \\
    生&いきる&いかす&&&いける&&「いきる」「いける」は、\kazu{\Rmnum{2}}の\kazu{3}\linebreak「いける」「いかす」は、\kazu{\Rmnum{4}}の他・他 \\
    懲&こりる&こらす&&&\kome こらしめる&& \\
    尽&つきる&つかす&&&つくす&&「つきる」「つくす」は、\kazu{\Rmnum{4}}の\kazu{4}\linebreak「つかす」「つくす」は、別語  \\
    閉&とじる&とざす&&&とじる&&「とじる」「とざす」は、\kazu{\Rmnum{4}}の他・他 \\
    伸&のびる&のばす&&&&& \\
    延&のびる&のばす&&&のべる&&「のびる」「のべる」は、\kazu{\Rmnum{2}}の\kazu{3}\linebreak「のばす」「のべる」は、\kazu{\Rmnum{4}}の他・他 \\
    満&みちる&みたす&&&&& \\
\end{hyo}

\pagebreak
\subsection{iru(自・上一段)・usu(他・サ行五段)\gosuu{2}}

\begin{hyo}*
    尽&つきる&つくす&&&つかす&&「つきる」「つかす」は、\kazu{\Rmnum{4}}の\kazu{3}\linebreak「つくす」「つかす」は、別語 \\
\end{hyo}

\subsection{iru(自・上一段)・osu(他・サ行五段)\gosuu{14}}

\begin{hyo}
    起&おきる&おこす&&おこる&&&「おこる」「おこす」は、\kazu{\Rmnum{1}}の\kazu{1}\linebreak「おきる」「おこる」は、\kazu{\Rmnum{4}}の自・自 \\
    落&おちる&おとす&&&&& \\
    降&おりる&おろす&&&&& \\
    下&おりる&おろす&&&&& \\
    過&すぎる&すごす&&&&& \\
    干&ひる&ほす&&&&& \\
    滅&ほろびる&ほろぼす&&&&& \\
\end{hyo}

\daashi

\subsection*{自動詞と自動詞の対立\gosuu{8}}

\begin{hyo}
    \SetCell[r=2]{c}起&\yama おきる&&&&\SetCell[r=2]{c}おこす&& \\
    &おこる&&&&&& \\ \pagebreak
    \SetCell[r=2]{c}転&\yama ころがる&&&\SetCell[r=2]{c}\kome ころぶ&\SetCell[r=2]{c}ころがす&& \\
    &ころげる&&&&&& \\
    \SetCell[r=2]{c}抜&\yama ぬける&&&&ぬかす&& \\
    &ぬかる&&&&ぬく&& \\
    \SetCell[r=2]{c}膨&\yama ふくれる&&&&&& \\
    &ふくらむ&&&&&& \\
\end{hyo}

\subsection*{他動詞と他動詞の対立\gosuu{8}}

\begin{hyo}
    \SetCell[r=2]{c}生&&\yama いかす&&\SetCell[r=2]{c}いきる&&& \\
    &&いける&&&&& \\
    \SetCell[r=2]{c}垂&&\yama たらす&&\SetCell[r=2]{c}たれる&&& \\
    &&たれる&&&&& \\
    \SetCell[r=2]{c}延&&\yama のばす&&\SetCell[r=2]{c}のびる&&& \\
    &&のべる&&&&& \\
    \SetCell[r=2]{c}閉&&\yama とざす&&\SetCell[r=2]{c}とじる&&& \\
    &&とじる&&&&& \\
\end{hyo}

\section[\jidori{4\zw}{延言}]{\jidori{4\zw}{延\ruby{言}{af}}\gosuu{\textbf{22}}}

\begin{enngenn}
    押&&おす&&&&おさえる&& \\
    語&&かたる&&かたらう&&&& \\
    住&すむ&&&すまう&&&& \\
    損&&&&そこなう&&そこねる&& \\
    捕&&とる&&&とらわれる&とらえる&& \\
    計&&はかる&&はからう&&&& \\
    恥&&はじる&&はじらう&&&& \\
    踏&&ふむ&&&&ふまえる&& \\
    振&&ふる&&ふるう&&&&\akigumi{1\zw}{音韻変化} \\
    向&むく&&&むかう&&むける&& \\
\end{enngenn}
