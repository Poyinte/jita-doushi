\section[活用形による対応]{活用形による対応\gosuu{\textbf{120}}}

\subsection{五段活用(自)・下一段活用(他)\gosuu{82}}

\begin{hyo}
  明&あく&あける&&あける&あかす&&「あく」「あかす」は、\kazu{\Rmnum{3}}の\kazu{1}\linebreak「あける」「あかす」は、\kazu{\Rmnum{4}}の\kazu{2}\\
  開&あく&あける&&&&& \\
  空&あく&あける&&&&& \\
  痛&いたむ&いためる&&&&& \\
  傷&いたむ&いためる&&&&& \\
  入&いる&いれる&&\kome はいる&&& \\
  浮&うかぶ&うかべる&&うく&&&\SetCell[r=2]{l}「うく」「うかれる」は、\kazu{\Rmnum{3}}の自・自 \\
  &&&&うかれる&&& \\
  傾&かたむく&かたむける&&&&& \\
  構&かまう&かまえる&&&かまう&&「かまう」「かまえる」は、\kazu{\Rmnum{2}}の他・他 \\
  苦&くるしむ&くるしめる&&&&& \\
  込&こむ&こめる&&&&& \\
  沈&しずむ&しずめる&&&&& \\
  従&したがう&したがえる&&&&& \\
  退&しりぞく&しりぞける&&&&& \\
  進&すすむ&すすめる&&&&& \\
  添&そう&そえる&&&&& \\
  育&そだつ&そだてる&&&&& \\
  背&そむく&そむける&&&&& \\
  建&たつ&たてる&&&&& \\
  立&たつ&たてつ&&&&& \\
  違&ちがう&ちがえる&&&&& \\
  縮&ちぢむ&ちぢめる&&ちぢまる&&&「ちぢまる」「ちぢめる」は、\kazu{\Rmnum{4}}の\kazu{1}\linebreak「ちぢむ」「ちぢまる」は、\kazu{\Rmnum{3}}の自・自 \\
  付&つく&つける&&&&& \\
  就&つく&つける&&&&& \\
  着&つく&つける&&&&& \\
  伝&つたう&つたえる&&つたわる&&&「つたわる」「つたえる」は、\kazu{\Rmnum{4}}の\kazu{1}\linebreak「つたう」「つたわる」は、\kazu{\Rmnum{3}}の自・自 \\
  続&つづく&つづける&&&&& \\
  詰&つむ&つめる&&つまる&&&\SetCell[r=2]{l}「つまる」「つめる」は、\kazu{\Rmnum{4}}の\kazu{1}\linebreak「つむ」「つめる」は、\kazu{\Rmnum{2}}の自・自\linebreak「つむ」「つまる」は、\kazu{\Rmnum{3}}の自・自 \\
  &&&&つめる&&& \\
  届&とどく&とどける&&&&& \\
  調&ととのう&ととのえる&&&&& \\
  整&ととのう&ととのえる&&&&& \\
  慰&なぐさむ&なぐさめる&&&&& \\
  懐&なつく&なつける&&&&& \\
  並&ならぶ&ならべる&&&&& \\
  潜&ひそむ&ひそめる&&&&& \\
  伏&ふす&すせる&&&&& \\
  向&むく&むける&&むかう&&&「むかう」は「むく」の延言 \\
  休&やすむ&やすめる&&やすまる&&&「やすまる」「やすめる」は、\kazu{\Rmnum{4}}の\kazu{1}\linebreak「やすむ」「やすまる」は、\kazu{\Rmnum{3}}の自・自 \\
  和&やわらぐ&やわらげる&&&&& \\
  緩&ゆるむ&ゆるめる&&&&& \\
\end{hyo}

\begin{reigai}
  陥&おちいる&\hspace{-5pt}おとしいれる&&&&&「おちいる」「おとしいれる」ともに複合語 \\
\end{reigai}

\subsection{下一段活用(自)・五段活用(他)\gosuu{34}}

\begin{hyo}
  売&うれる&うる&&&&& \\
  折&おれる&おる&&&&& \\
  欠&かける&かく&&&&& \\
  切&きれる&きる&&&&& \\
  砕&くだける&くだく&&&&& \\
  裂&さける&さく&&&&& \\
  擦&すれる&する&&&&& \\
  解&とける&とく&&&とかす&&「とける」「とかす」は、\kazu{\Rmnum{4}}の\kazu{2}\linebreak「とく」「とかす」は、\kazu{\Rmnum{3}}の他・他 \\
  溶&とける&とく&&&とかす&&「とける」「とかす」は、\kazu{\Rmnum{4}}の\kazu{2}\linebreak「とく」「とかす」は、\kazu{\Rmnum{3}}の他・他 \\
  抜&ぬける&ぬく&&ぬかる&ぬかす&&「ぬかる」「ぬく」は、\kazu{\Rmnum{3}}の\kazu{3}\linebreak「ぬける」「ぬかす」は、\kazu{\Rmnum{4}}の\kazu{2}\linebreak「ぬく」「ぬかす」は、\kazu{\Rmnum{3}}の他・他\linebreak「ぬける」「ぬかる」は、\kazu{\Rmnum{4}}の自・自 \\
  脱&ぬげる&ぬぐ&&&&& \\
  引&ひける&ひく&&ひく&&&「ひく」「ひける」は、\kazu{\Rmnum{2}}の自・自 \\
  開&ひらける&ひらく&&ひらく&&&「ひらく」「ひらける」は、\kazu{\Rmnum{2}}の自・自 \\
  焼&やける&やく&&&&& \\
  破&やぶれる&やぶる&&&&& \\
  \SetCell[r=2]{c}揺&\SetCell[r=2]{c}ゆれる&\SetCell[r=2]{c}ゆる&&\kome ゆらぐ&&& \\
  &&&&\kome ゆるく&&& \\
  割&われる&わる&&&&& \\
\end{hyo}

\pagebreak
\subsection{上一段活用(自)・下一段活用(他)\gosuu{4}}

\begin{hyo}
  生&いきる&いける&&&いかす&&「いきる」「いかす」は、\kazu{\Rmnum{4}}の\kazu{3}\linebreak「いける」「いかす」は、\kazu{\Rmnum{4}}の他・他 \\
  延&のびる&のべる&&&のばす&&「のびる」「のばす」は、\kazu{\Rmnum{4}}の\kazu{3}\linebreak「のべる」「のばす」は、\kazu{\Rmnum{4}}の他・他 \\
\end{hyo}

\daashi

\subsection*{自動詞と自動詞の対立\gosuu{20}}

\begin{hyo}
  \SetCell[r=2]{c}腐&\yama くさる&&&&\SetCell[r=2]{c}くさらす&&以下「わかる・わかれる」まで\kazu{18}語は、五段活用と下一段活用で対立 \\
  &くされる&&&&&& \\
  \SetCell[r=2]{c}透&\yama すく&&&&\SetCell[r=2]{c}すかす&& \\
  &すける&&&&&& \\
  \SetCell[r=2]{c}廃&\yama すたる&&&&&& \\
  &すたれる&&&&&& \\
  \SetCell[r=2]{c}詰&\yama つむ&&&\SetCell[r=2]{c}つまる&\SetCell[r=2]{c}つめる&& \\
  &つめる&&&&&& \\
  \SetCell[r=2]{c}引&\yama ひく&&&&ひく&& \\
  &ひける&&&&&& \\
  \SetCell[r=2]{c}開&\yama ひらく&&&&\SetCell[r=2]{c}ひらく&& \\
  &ひらける&&&&&& \\
  \SetCell[r=2]{c}震&\yama ふるう&&&&&& \\
  &ふるえる&&&&&& \\
  \SetCell[r=2]{c}漏&\yama もる&&&&\SetCell[r=2]{c}もらす&& \\
  &もれる&&&&&& \\ \pagebreak
  \SetCell[r=2]{c}分&\yama わかる&&&&わける&& \\
  &わかれる&&&&\kome わかつ&& \\
  \SetCell[r=2]{c}足&\yama たる&&&&\SetCell[r=2]{c}たす&&\SetCell[r=2]{l}五段活用と上一段活用で対立 \\
  &たりる&&&&&& \\
\end{hyo}

\subsection*{他動詞と他動詞の対立\gosuu{10}}

\begin{hyo}
  \SetCell[r=2]{c}合&&\yama あわす&&\SetCell[r=2]{c}あう&&&\SetCell[r=2]{l}以下「ふくむ・ふくめる」まで、五段活用と下一段活用で対立 \\
  &&あわせる&&&&& \\
  \SetCell[r=2]{c}卑&&\yama いやしむ&&&&& \\
  &&いやしめる&&&&& \\
  \SetCell[r=2]{c}構&&\yama かまう&&\SetCell[r=2]{c}かまう&&& \\
  &&かまえる&&&&& \\
  \SetCell[r=2]{c}上&&\yama のぼす
  &&\SetCell[r=2]{c}のぼる&&& \\
  &&のぼせる&&&&& \\
  \SetCell[r=2]{c}含&&\yama ふくむ&&&&& \\
  &&ふくめる&&&&& \\
\end{hyo}

\subsection*{使役動詞と使役動詞の対立\gosuu{2}}

\begin{hyo}
  \SetCell[r=2]{c}任&&&\yama まかす&&&&\SetCell[r=2]{l}五段活用と下一段活用で対立 \\
  &&&まかせる&&&& \\
\end{hyo}