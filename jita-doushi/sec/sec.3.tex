\section[基本形とその派生形の対応]{基本形とその派生形の対応\gosuu{\textbf{66}}}

\subsection{u(自・五段)・asu(他・サ行五段)\gosuu{48}}

\begin{hyo}
    合&あう&あわす&&&あわせる&&「あう」「あわせる」は、\kazu{\Rmnum{3}}の\kazu{6}\linebreak「あわす」「あわせる」は、\kazu{\Rmnum{2}}の他・他 \\
    明&あく&あかす&&あける&あける&&「あく」「あける」は、\kazu{\Rmnum{2}}の\kazu{1}\linebreak「あける」「あかす」は、\kazu{\Rmnum{4}}の\kazu{2} \\
    動&うごく&うごかす&&&&& \\
    驚&おどろく&おどろかす&&&&& \\
    交&かう&かわす&&&&& \\
    乾&わかく&かわかす&&&&& \\
    腐&くさる&くさらす&&くされる&&&「くされる」「くさらす」は、\kazu{\Rmnum{4}}の\kazu{2}\linebreak「くさる」「くされる」は、\kazu{\Rmnum{2}}の自・自 \\
    凝&こる&こらす&&&&& \\
    透&すく&すかす&&すける&&&「すける」「すかす」は、\kazu{\Rmnum{4}}の\kazu{2}\linebreak「すく」「すける」は、\kazu{\Rmnum{2}}の自・自 \\
    済&すむ&すます&&&&& \\
    澄&すむ&すます&&&&& \\
    反&そる&そらす&&&&& \\
    散&ちる&ちらす&&ちらかる&ちらかす&&「ちらかる」「ちらかす」は、\kazu{\Rmnum{1}}の\kazu{1} \\
    遣&&つかう&つかわす&&&&「他使の対応」、形態上ここに整理した。 \\
    照&てる&てらす&&てれる&&&「てれる」「てらす」は、\kazu{\Rmnum{4}}の\kazu{2}\linebreak「てる」「てれる」は別語 \\
    飛&とぶ&とばす&&&&& \\
    悩&なやむ&なやます&&&&& \\
    鳴&なる&ならす&&&&& \\
    励&はげむ&はげます&&&&& \\
    減&へる&へらす&&&&& \\
    蒸&むす&むらす&&むれる&むす&&「むす」「むらす」は、語根が異なる。\linebreak「むれる」「むす」は、\kazu{\Rmnum{1}}の\kazu{2}\linebreak「むれる」「むらす」は、\kazu{\Rmnum{4}}の\kazu{2}\linebreak「むす」「むらす」は、\kazu{\Rmnum{3}}の他・他\linebreak「むす」「むれる」は、\kazu{\Rmnum{1}}の自・自 \\
    漏&もる&もらす&&もれる&&&「もれる」「もらす」は、\kazu{\Rmnum{4}}の\kazu{2}\linebreak「もる」「もれる」は、\kazu{\Rmnum{2}}の自・自 \\
    沸&わく&わかす&&&&& \\
    煩&わずらう&わずらわす&&&&& \\
\end{hyo}

\subsection{u(自・五段)・osu(他・サ行五段)\gosuu{2}}

\begin{hyo}*
    及&およぶ&およぼす&&&&& \\
\end{hyo}

\subsection{aru(自・ラ行五段)・u(他・五段)\gosuu{6}}

\begin{hyo}
    絡&からまる&からむ&&からむ&&&「からむ」「からまる」は、\kazu{\Rmnum{3}}の自・自 \\
    刺&ささる&さす&&&&& \\
    抜&ぬかる&ぬく&&ぬける&ぬかす&&「ぬける」「ぬく」は、\kazu{\Rmnum{2}}の\kazu{2}\linebreak「ぬける」「ぬかす」は、\kazu{\Rmnum{4}}の\kazu{2}\linebreak「ぬく」「ぬかす」は、\kazu{\Rmnum{3}}の他・他\linebreak「ぬかる」「ぬける」は、\kazu{\Rmnum{4}}の自・自 \\
\end{hyo}

\pagebreak
\subsection{oru(自・ラ行五段)・u(他・五段)\gosuu{2}}

\begin{hyo}*
    積&つもる&つむ&&&&& \\
\end{hyo}

\subsection{eru(自・下一段)・u(他・五段)\gosuu{6}}

\begin{hyo}
    生&うまれる&うむ&&&&& \\
    産&うまれる&うむ&&&&& \\
    聞&きこえる&きく&&&&& \\
\end{hyo}

\subsection{u(自・下一段)・seru(他・サ行下一段)\gosuu{2}}

\begin{hyo}*
    合&あう&あわせる&&&あわす&&「あう」「あわす」は、\kazu{\Rmnum{3}}の\kazu{1}\linebreak「あわす」「あわせる」は、\kazu{\Rmnum{2}}の他・他 \\
\end{hyo}

\daashi

\pagebreak
\subsection*{自動詞と自動詞の対立\gosuu{12}}

\begin{hyo}
    \SetCell[r=2]{c}浮&\yama うく&&&\SetCell[r=2]{c}うかぶ&\SetCell[r=2]{c}うかべる&&\SetCell[r=2]{l}「u・eru」の形態 \\
    &うかれる&&&&&& \\
    \SetCell[r=2]{c}絡&\yama からむ&&&&\SetCell[r=2]{c}からむ&& \\
    &からまる&&&&&&\SetCell[r=2]{l}以下「やすむ・やすまる」まで\kazu{10}語「u・eru」の形態の形態 \\
    \SetCell[r=2]{c}縮&\yama ちぢむ&&&&\SetCell[r=2]{c}ちぢめる&& \\
    &ちぢまる&&&&&& \\
    \SetCell[r=2]{c}伝&\yama つたう&&&&\SetCell[r=2]{c}つたえる&& \\
    &つたわる&&&&&& \\
    \SetCell[r=2]{c}詰&\yama つむ&&&\SetCell[r=2]{c}つめる&\SetCell[r=2]{c}つめる&& \\
    &つまる&&&&&& \\
    \SetCell[r=2]{c}休&\yama やすむ&&&&\SetCell[r=2]{c}やすめる&& \\
    &やすまる&&&&&& \\
\end{hyo}

\subsection*{他動詞と他動詞の対立\gosuu{12}}

\begin{hyo}
    \SetCell[r=2]{c}忍&&\yama しのぶ&&\SetCell[r=2]{c}しのぶ&&&\SetCell[r=2]{l}「u・seru」の形態 \\
    &&しのばせる&&&&& \\
    \SetCell[r=2]{c}解&&\yama とく&&\SetCell[r=2]{c}とける&&&\SetCell[r=2]{l}以下「もす・もやす」もで\kazu{10}語は「u・seru」の形態 \\
    &&とかす&&&&& \\
    \SetCell[r=2]{c}溶&&\yama とく&&\SetCell[r=2]{c}とける&&& \\
    &&とかす&&&&& \\ \pagebreak
    \SetCell[r=2]{c}抜&&\yama ぬく&&\SetCell[r=2]{c}ぬかる&\SetCell[r=2]{c}ぬける&& \\
    &&ぬかす&&&&& \\
    \SetCell[r=2]{c}蒸&&\yama むす&&むれる&&&\SetCell[r=2]{l}「むす」「むらす」は、語根が異なる。 \\
    &&むらす&&むす&&& \\
    \SetCell[r=2]{c}燃&&\yama もす&&\SetCell[r=2]{c}もえる&&&\SetCell[r=2]{l}「もやす」の派生の基は「もゆ」 \\
    &&もやす&&&&& \\
\end{hyo}