\chapter{\hspace{-2\zw}『当用漢字音訓表』の音訓欄に掲げられた動詞の「自他の対応」}

\vspace{-2.5\zw}
\begin{flushright}
\large(※印は、対応関係・対立関係を持たないと整理した動詞)\hspace*{1\zw}
\end{flushright}

\section[活用語尾による対応]{活用語尾による対応\gosuu{\textbf{128}}}

\subsection{\ruby{る}{ru}(自・ラ行五段)・\ruby{す}{su}(他・サ行五段)\gosuu{58}}

\begin{hyo}
  余&あまる&あます&&&&&\\
  \yama 至&いたる&&&&&&\SetCell[r=2]{l}「自他の対応」であるが、使用する漢字は異なる。\\
  致&&いたす&&&&&\\
  写&うつる&うつす&&&&& \\
  映&うつる&うつす&&&&& \\
  移&うつる&うつす&&&&& \\
  興&おこる&おこす&&&&& \\
  起&おこる&おこす&&おきる&&&「おきる」「おこす」は、\kazu{\Rmnum{4}}の\kazu{5}\linebreak「おきる」「おこる」は、\kazu{\Rmnum{4}}の自・自 \\
  帰&かえる&かえす&&&&& \\
  返&かえる&かえす&&&&& \\
  来&きたる&きたす&&\kome くる&&& \\
  下&くだる&くだす&&&&\kome くださる& \\
  覆&くつがえる&くつがえす&&&&& \\
  転&ころがる&ころがす&&\kome ころぶ&&&\SetCell[r=2]{l}「ころげる」「ころがす」は、\kazu{\Rmnum{4}}の\kazu{2}\linebreak「ころげる」「ころがる」は、\kazu{\Rmnum{4}}の自・自 \\
  &&&&ころげる&&& \\
  \yama 悟&&さとる&&&&&\SetCell[r=2]{l}「自他の対応」、形態上、ここに整理した。 \\
  諭&&&さとす&&&& \\
  湿&しめる&しめす&&&&& \\
  足&たる&たす&&たりる&&&「たりる」「たす」は、\kazu{\Rmnum{1}}の\kazu{4}\linebreak「たる」「たりる」は、\kazu{\Rmnum{2}}の自・自 \\
  散&ちらかる&ちらかす&&ちる&ちらす&&「ちる」「ちらす」は、\kazu{\Rmnum{3}}の\kazu{1} \\
  通&とおる&とおす&&&&& \\
  治&なおる&なおす&&&&& \\
  直&なおる&なおす&&&&& \\
  成&なる&なす&&&&& \\
  濁&にごる&にごす&&&&& \\
  残&のこる&のこす&&&&& \\
  上&のぼる&のぼす&&&のぼせる&&「のぼる」「のぼせる」は、\kazu{\Rmnum{1}}の\kazu{3}\linebreak「のぼす」「のぼせる」は、\kazu{\Rmnum{2}}の他・他 \\
  浸&ひたる&ひたす&&&&& \\
  翻&ひるがえる&ひるがえす&&&&& \\
  回&まわる&まわす&&&&& \\
  宿&やどる&やどす&&&&& \\
  渡&わたる&わたす&&&&& \\
\end{hyo}

\subsection{\ruby{れる}{reru}(自・ラ行下一段)・\ruby{す}{su}(他・サ行五段)\gosuu{32}}

\begin{hyo}
  表&あらわれる&あらわす&&&&& \\
  現&あらわれる&あらわす&&&&& \\
  隠&かくれる&かくす&&&&& \\
  崩&くずれる&くずす&&&&& \\
  汚&けがれる&けがす&&&&& \\
  焦&こがれる&こがす&&こげる&&&「こげる」「こがす」は、\kazu{\Rmnum{4}}の\kazu{2}\linebreak「こげる」「こがれる」は、別語 \\
  壊&こわれる&こわす&&&&& \\
  流&ながれる&ながす&&&&& \\
  逃&のがれる&のがす&&にげる&にがす&&「にげる」「にがす」は、\kazu{\Rmnum{4}}の\kazu{2} \\
  外&はずれる&はずす&&&&& \\
  離&はなれる&はなす&&&&& \\
  放&はなれる&はなす&&&\kome はなつ&& \\
  蒸&むれる&むす&&むす&むらす&&「むす」「むらす」は、\kazu{\Rmnum{3}}の\kazu{1}\linebreak「むれる」「むらす」は、\kazu{\Rmnum{4}}の\kazu{2}\linebreak「むす」「むらす」は、\kazu{\Rmnum{3}}の他・他\linebreak「むす」「むれる」は、\kazu{\Rmnum{1}}の自・自 \\
  汚&よごれる&&&&&& \\
\end{hyo}

\subsection{\ruby{る}{ru}(自・ラ行五段)・\ruby{せる}{seru}(他・サ行下一段)\gosuu{8}}

\begin{hyo}
  上&のぼる&のぼせる&&&のぼす&&「のぼる」「のぼす」は、\kazu{\Rmnum{1}}の\kazu{1}\linebreak「のぼす」「のぼせる」は、\kazu{\Rmnum{2}}の他・他 \\
  乗&のる&のせる&&&&& \\
  載&のる&のせる&&&&& \\
  寄&よる&よせる&&よせる&&&「よる」「よせる」は、\kazu{\Rmnum{1}}の自・自 \\
\end{hyo}

\subsection{\ruby{りる}{riru}(自・ラ行上一段)・\ruby{す}{su}(他・サ行五段)\gosuu{4}}

\begin{hyo}
  足&たりる&たす&&たる&&&「たる」「たす」は、\kazu{\Rmnum{1}}の\kazu{1}\linebreak「たる」「たりる」は、\kazu{\Rmnum{2}}の自・自 \\
  \yama 借&&かりる&&&&&\SetCell[r=2]{l}「他使の対応」、形態上ここに整理した。 \\
  貸&&&かす&&&& \\
\end{hyo}

\pagebreak
\subsection{\ruby{エる}{eru}(自・ナ行下一段)・\ruby{す}{su}(他・サ行五段)\gosuu{2}}

\begin{hyo}*
  寝&ねる&ねかす&&&&&語根が異なる。 \\
\end{hyo}

\subsection{\ruby{える}{eru}(自・ラ行下一段)・\ruby{す}{su}(他・サ行五段)\gosuu{8}}

\begin{hyo}
  消&きえる&けす&&&&&語根が異なる。 \\
  越&こえる&こす&&&&& \\
  超&こえる&こす&&&&& \\
  燃&もえる&もす&&&もやす&&「もえる」「もやす」は、\kazu{\Rmnum{4}}の\kazu{2}\linebreak「もす」「もやす」は、\kazu{\Rmnum{3}}の他・他 \\
\end{hyo}

\subsection{\ruby{える}{}eru(自・ラ行下一段)・\ruby{イる}{iru}(他・上一段)\gosuu{4}}

\begin{hyo}
  煮&にえる&にる&&&&& \\
  見&みえる&みる&&&&みせる&「みえる」「みせる」は、\kazu{\Rmnum{1}}の\kazu{8}\linebreak「みる」「みせる」は、\kazu{\Rmnum{1}}の\kazu{11} \\
\end{hyo}

\subsection{\ruby{える}{eru}(自・ア行下一段)・\ruby{せる}{seru}(他・サ行下一段)\gosuu{2}}

\begin{hyo}*
  見&みえる&&みせる&&みる&&「みえる」「みる」は、\kazu{\Rmnum{1}}の\kazu{7}\linebreak「みる」「みせる」は、\kazu{\Rmnum{1}}の\kazu{11} \\
\end{hyo}

\subsection{\ruby{う}{u}(自・五段)・\ruby{す}{su}(他・サ行五段)\gosuu{2}}

\begin{hyo}*
  潤&うるおう&うるおす&&\kome うるむ&&& \\
\end{hyo}

\subsection{\ruby{る}{ru}(自・五段)・\ruby{える}{eru}(他・ア行下一段)\gosuu{2}}

\begin{hyo}*
  捕&つかまる&つかまえる&&&&& \\
\end{hyo}

\subsection{\ruby{イる}{iru}(他・上一段)・\ruby{せる}{seru}(使・サ行下一段)\gosuu{6}}

\begin{hyo}
  浴&&あびる&あびせる&&&&以下「みる」「みせる」まで\kazu{6}語「他使の対応」 \\
  着&&きる&きせる&&&& \\
  見&&みる&みせる&みえる&&&「みえる」「みる」は、\kazu{\Rmnum{1}}の\kazu{7}\linebreak「みえる」「みせる」は、\kazu{\Rmnum{1}}の\kazu{8} \\
\end{hyo}

\daashi

\subsection*{自動詞と自動詞の対立\gosuu{4}}

\begin{hyo}
  \SetCell[r=2]{c}蒸&\yama むす&&&&むす&& \\
  &むれる&&&&むらす&& \\
  \SetCell[r=2]{c}寄&\yama よる&&&&よせる&& \\
  &よせる&&&&&& \\
\end{hyo}